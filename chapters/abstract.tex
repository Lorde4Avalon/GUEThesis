% !Mode:: "TeX:UTF-8"


\begin{chineseabstract}

摘要应概括论文的主要信息,应具有独立性和自含性,即不阅读论文的全文,就能获得必要的信息。摘要内容一般应包括研究目的、内容、方法、成果和结论,要突出论文的创造性成果或新见解,不要与绪论相混淆。语言力求精练、准确,以300-500 字为宜。关键词是供检索用的主题词条,应体现论文特色,具有语义性,在论文中有明确的出处,并应尽量采用《汉语主题词表》或各专业主题词表提供的规范词。关键词与摘要应在同一页,在摘要的下方另起一行注明,一般列 3-5 个,按词条的外延层次排列(外延大的排在前面)。

\chinesekeyword{(关键词一般为5个左右,内容采用小四号、宋体、接排、各关键词之间用分号隔开)}
\end{chineseabstract}

\begin{englishabstract}

The content of the English abstract is the same as the Chinese abstract, 250-400 content words are appropriate. Start another line below the abstract to indicate English.

\englishkeyword{(Keywords 3-5 各关键词之间用分号隔开)}
\end{englishabstract}