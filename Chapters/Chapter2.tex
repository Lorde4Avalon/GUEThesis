% !Mode:: "TeX:UTF-8"
%此为章节二模板
%\chapter、\section、\subsection、\subsubsection分别对应一二三四级标题
\chapter{表格示例}\label{ch:2}

可使用excel绘制表格,然后粘贴到以下网站中生成latex表格代码。

推荐网站如下:

https://www.tablesgenerator.com/

https://www.latex-tables.com/

\section{普通三线表示例}
普遍学者认为,微通道指的是水力直径在 $10\ \rm{\mu m}$ 到 $1000\ \rm{\mu m}$ 范围内的通道(也有观点认为是 $1\ \rm{\mu m}$ 到 $100\ \rm{\mu m}$)所构成的换热器。
以下是较为常见的微通道尺寸分类,可以参见\cref{tab:division-of-microchannels}。
\begin{table}[htbp]
    \caption[微通道的划分]{微通道的划分\cite{LuSiHong_2021}}
    \setlength{\tabcolsep}{14mm}{ % 因表格过窄,手动设置宽度为7mm
        \begin{tabular}{lc}
            \toprule
            通道种类    & 水力直径$\mu m$   \\
            \midrule
            分子纳米通道  & $\le 0.1$     \\
            过渡性纳米通道 & $0.1\sim 1$   \\
            过渡性微通道  & $1\sim 10$    \\
            微通道     & $10\sim 1000$ \\
            常规通道    & $>1000$       \\
            \bottomrule
        \end{tabular}}
    \label{tab:division-of-microchannels}
\end{table}



\section{跨页表格示例}

\begin{longtable}{@{\extracolsep{\fill}}cccccc@{}}  \\
    \caption{RSM仿真实验规划表}
    \label{tab:Experimental-Planning}  \\
    \toprule
    标准序 & 运行序 & $H_{rib}\ \rm{(mm)}$ & $H_{pf}\ \rm{(mm)}$ & $N_{pf}$ & $N_{ac}$ \\ \midrule
    \endfirsthead
    %
    \multicolumn{6}{c}%
    {{表 \thetable\ RSM仿真实验规划表 (续)}} \\
    \toprule
    标准序 & 运行序 & $H_{rib}\ \rm{(mm)}$ & $H_{pf}\ \rm{(mm)}$ & $N_{pf}$ & $N_{ac}$ \\ \midrule
    \endhead
    %
    \bottomrule
    \endfoot
    %
    \endlastfoot
    %
    11  & 1   & 0.16            & 0.8            & 6        & 16       \\
    13  & 2   & 0.16            & 0.16           & 22       & 16       \\
    15  & 3   & 0.16            & 0.8            & 22       & 16       \\
    12  & 4   & 0.8             & 0.8            & 6        & 16       \\
    10  & 5   & 0.8             & 0.16           & 6        & 16       \\
    2   & 6   & 0.8             & 0.16           & 6        & 0        \\
    19  & 7   & 0.48            & 0.48           & 14       & 8        \\
    1   & 8   & 0.16            & 0.16           & 6        & 0        \\
    20  & 9   & 0.48            & 0.48           & 14       & 8        \\
    18  & 10  & 0.48            & 0.48           & 14       & 8        \\
    8   & 11  & 0.8             & 0.8            & 22       & 0        \\
    14  & 12  & 0.8             & 0.16           & 22       & 16       \\
    6   & 13  & 0.8             & 0.16           & 22       & 0        \\
    17  & 14  & 0.48            & 0.48           & 14       & 8        \\
    7   & 15  & 0.16            & 0.8            & 22       & 0        \\
    16  & 16  & 0.8             & 0.8            & 22       & 16       \\
    4   & 17  & 0.8             & 0.8            & 6        & 0        \\
    9   & 18  & 0.16            & 0.16           & 6        & 16       \\
    5   & 19  & 0.16            & 0.16           & 22       & 0        \\
    3   & 20  & 0.16            & 0.8            & 6        & 0        \\
    25  & 21  & 0.48            & 0.48           & 6        & 8        \\
    22  & 22  & 0.8             & 0.48           & 14       & 8        \\
    23  & 23  & 0.48            & 0.16           & 14       & 8        \\
    29  & 24  & 0.48            & 0.48           & 14       & 8        \\
    28  & 25  & 0.48            & 0.48           & 14       & 16       \\
    30  & 26  & 0.48            & 0.48           & 14       & 8        \\
    26  & 27  & 0.48            & 0.48           & 22       & 8        \\
    27  & 28  & 0.48            & 0.48           & 14       & 0        \\
    21  & 29  & 0.16            & 0.48           & 14       & 8        \\
    24  & 30  & 0.48            & 0.8            & 14       & 8        \\ \bottomrule
\end{longtable}

\section{本章小节}
本章介绍了基于嵌入式散热模块的微通道散热技术所涉及的基……